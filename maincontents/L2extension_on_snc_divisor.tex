% !TEX root = ../chatGPT.tex
\chapter{\texorpdfstring{$L^2$}{}-Extension Theorem in the case of singular subvariety with simple normal crossing divisor}
\section{The Case of a Non-singular Hypersurface \texorpdfstring{$Y$}{}}
\begin{problem}[Computation Problem 1]\label{prob:cp1}
    Let $\xi$ be a $L$-valued form of type $(n, 1)$. We denote by $\gamma_{\xi}:=\star \xi$ its Hodge dual (induced by an arbitrary Kähler form on $X$ ).
Consider the functional
\begin{equation}
    \label{eq:L1}
    \mathcal{F}(\xi)=\int_X \bar{\partial}\left(\frac{u}{s_Y}\right) \wedge \overline{\gamma_{\xi}} e^{-\varphi_L} 
\end{equation}
associated to the current $\bar{\partial}\left(\frac{u}{s_Y}\right)$.
We decompose $\xi=\xi_1+\xi_2$ according to $\operatorname{Ker}(\bar{\partial})$ and $\operatorname{Ker}(\bar{\partial})^{\perp}$. It turns out that we have the equality
\begin{equation}
    \label{eq:L2}
    \mathcal{F}(\xi)=\int_Y \frac{u}{\mathrm{~d} s_Y} \wedge \overline{\gamma_{\xi_1}} e^{-\varphi_L}
\end{equation}
请问如何从\eqref{eq:L1}得到\eqref{eq:L2}?
\end{problem}
    
\begin{solution}
    要从\eqref{eq:L1}得到\eqref{eq:L2},我们需要利用两个关键点:\textbf{形式的分解和全纯微分算子的性质}。
\begin{enumerate}
\item  \textbf{形式的分解}

根据题意,形式 $\xi$ 可以分解为两部分:
$$
\xi = \xi_1 + \xi_2
$$
其中 $\xi_1 \in \operatorname{Ker}(\bar{\partial})$,即 $\bar{\partial} \xi_1 = 0$;$\xi_2 \in \operatorname{Ker}(\bar{\partial})^{\perp}$,即 $\xi_2$ 是 $\operatorname{Ker}(\bar{\partial})$ 在某种内积下的正交补。

\item  \textbf{Hodge 对偶}

由 $\xi$ 的 Hodge 对偶 $\gamma_{\xi} = \star \xi$ 的定义,我们也可以相应地分解 $\gamma_{\xi}$ 为:
$$
\gamma_{\xi} = \gamma_{\xi_1} + \gamma_{\xi_2}
$$

\item  \textbf{运算的性质}

首先,我们注意到 $\mathcal{F}(\xi)$ 的定义:
$$
\mathcal{F}(\xi) = \int_X \bar{\partial}\left(\frac{u}{s_Y}\right) \wedge \overline{\gamma_{\xi}} e^{-\varphi_L}
$$

接下来,考虑 $\bar{\partial}$ 作用到分解形式 $\xi$ 上的性质。由于 $\xi_1 \in \operatorname{Ker}(\bar{\partial})$,所以 $\bar{\partial} \xi_1 = 0$,而 $\xi_2 \in \operatorname{Ker}(\bar{\partial})^{\perp}$ 意味着 $\xi_2$ 与 $\operatorname{Ker}(\bar{\partial})$ 正交。

\item  \textbf{计算 $\mathcal{F}(\xi)$}

由于 $\xi_2 \in \operatorname{Ker}(\bar{\partial})^{\perp}$,它不包含在 $\operatorname{Ker}(\bar{\partial})$ 中,意味着 $\bar{\partial}(\xi_2)$ 不为零。然而,由于 $\bar{\partial}$ 是一个偏微分算子,我们需要考虑其在积分中的作用。

令我们考虑当前 $\bar{\partial}(\xi_1 + \xi_2)$ 在积分中的影响:

$$
\mathcal{F}(\xi) = \int_X \bar{\partial}\left(\frac{u}{s_Y}\right) \wedge \overline{(\gamma_{\xi_1} + \gamma_{\xi_2})} e^{-\varphi_L}
$$

由于 $\bar{\partial} \xi_1 = 0$,我们有:
$$
\mathcal{F}(\xi) = \int_X \bar{\partial}\left(\frac{u}{s_Y}\right) \wedge \overline{\gamma_{\xi_1}} e^{-\varphi_L} + \int_X \bar{\partial}\left(\frac{u}{s_Y}\right) \wedge \overline{\gamma_{\xi_2}} e^{-\varphi_L}
$$

\textbf{关键点在于 $\gamma_{\xi_2}$ 项。因为 $\xi_2$ 与 $\operatorname{Ker}(\bar{\partial})$ 正交,所以它的贡献可以被看作零。换句话说,我们可以忽略 $\gamma_{\xi_2}$ 相关的积分项。}故此:
$$
\mathcal{F}(\xi) = \int_X \bar{\partial}\left(\frac{u}{s_Y}\right) \wedge \overline{\gamma_{\xi_1}} e^{-\varphi_L}
$$

\item  \textbf{积分限制到 $Y$ 上}

接下来考虑 $\int_X \bar{\partial}\left(\frac{u}{s_Y}\right) \wedge \overline{\gamma_{\xi_1}} e^{-\varphi_L}$ 的计算。由于 $\bar{\partial}\left(\frac{u}{s_Y}\right)$ 是由一个除子 $Y$ 产生的全纯形式,我们可以把积分限制到 $Y$ 上。

特别是,\textbf{$\bar{\partial}\left(\frac{u}{s_Y}\right)$ 在 $Y$ 上有一个“delta-like”的表现形式,它可以描述为沿 $Y$ 的拉回。}所以我们有:
$$
\int_X \bar{\partial}\left(\frac{u}{s_Y}\right) \wedge \overline{\gamma_{\xi_1}} e^{-\varphi_L} = \int_Y \frac{u}{\mathrm{d} s_Y} \wedge \overline{\gamma_{\xi_1}} e^{-\varphi_L}
$$
\end{enumerate}
综上所述,通过上述分解和性质的利用,我们得到:
$$
\mathcal{F}(\xi) = \int_Y \frac{u}{\mathrm{d} s_Y} \wedge \overline{\gamma_{\xi_1}} e^{-\varphi_L}
$$

这样我们就从\eqref{eq:L1}得到\eqref{eq:L2}。
\end{solution}

\begin{remark}
    这两个问题涉及到一些深层次的复几何和代数几何的理论。我们来一一解答:

1. 为何从 $\xi_2$ 与 $\operatorname{Ker}(\bar{\partial})$ 正交可以得出它的贡献可以被看作零?

$\xi_2$ 属于 $\operatorname{Ker}(\bar{\partial})^{\perp}$,即它与 $\operatorname{Ker}(\bar{\partial})$ 中的所有元素正交。为了理解这一点,我们考虑如下几点:
\begin{itemize}
    \item 在积分 $\mathcal{F}(\xi) = \int_X \bar{\partial}\left(\frac{u}{s_Y}\right) \wedge \overline{\gamma_{\xi}} e^{-\varphi_L}$ 中,我们可以将 $\gamma_{\xi}$ 分解为 $\gamma_{\xi_1}$ 和 $\gamma_{\xi_2}$。
    \item $\xi_1$ 在 $\operatorname{Ker}(\bar{\partial})$ 中,这意味着 $\bar{\partial} \xi_1 = 0$。而 $\xi_2 \in \operatorname{Ker}(\bar{\partial})^{\perp}$,意味着 $\xi_2$ 与所有 $\operatorname{Ker}(\bar{\partial})$ 中的元素正交。
    \item 当我们计算 $\mathcal{F}(\xi)$ 时,由于 $\bar{\partial}\left(\frac{u}{s_Y}\right)$ 是一个 $\bar{\partial}$-闭的 $(n, 1)$ 形式,其支撑集在 $Y$ 上。$\gamma_{\xi_2}$ 与 $\gamma_{\xi_1}$ 是正交的,而 $\gamma_{\xi_1}$ 与 $\bar{\partial}\left(\frac{u}{s_Y}\right)$ 的结合在积分中起主要作用。
\end{itemize}

由于 $\bar{\partial} \xi_1 = 0$,$\xi_1$ 的贡献可以直接用积分表示出来。对于 $\xi_2$,因为它与 $\operatorname{Ker}(\bar{\partial})$ 正交,它的作用在积分中是零。这样我们就可以忽略 $\gamma_{\xi_2}$ 的相关积分。

总结:$\xi_2$ 与 $\operatorname{Ker}(\bar{\partial})$ 正交的性质,使得在涉及 $\bar{\partial}$ 的积分中,$\xi_2$ 的贡献是零。

2. $\bar{\partial}\left(\frac{u}{s_Y}\right)$ 在 $Y$ 上有一个“delta-like”的表现形式,它可以描述为沿 $Y$ 的拉回?

这里的关键是理解除子和其相关的全纯结构。
\begin{itemize}
    \item \textbf{Delta-like 表现形式}

在复几何中,如果 $Y$ 是 $X$ 的一个除子(即 $Y$ 是 $X$ 中的一个子流形,并且它可以作为一个零点集定义),我们可以考虑如下形式:
$$
\bar{\partial}\left(\frac{1}{s_Y}\right)
$$
这里 $s_Y$ 是定义 $Y$ 的一个全纯函数。$s_Y = 0$ 在 $Y$ 上定义了 $Y$。我们通过分式表达 $\frac{1}{s_Y}$ 来描述 $Y$ 的某些解析性质。

这种形式 $\bar{\partial}\left(\frac{1}{s_Y}\right)$ 可以理解为一个“delta-like” 表现形式,因为它在 $Y$ 上表现为一个集中在 $Y$ 上的电流(current)。这种形式的积分性质类似于 Dirac delta 函数在 $Y$ 上的限制。
\item \textbf{沿 $Y$ 的拉回}

考虑形式 $\bar{\partial}\left(\frac{u}{s_Y}\right)$,其中 $u$ 是一个 $L$-值的形式。我们可以将 $u$ 限制在 $Y$ 上,通过 $\frac{u}{s_Y}$ 体现出 $u$ 在 $Y$ 上的行为。

这种“拉回”(pullback)意味着我们在计算 $\mathcal{F}(\xi)$ 的时候,可以将积分集中在 $Y$ 上的一个截面,这样就转换为沿 $Y$ 的积分。
\end{itemize}

总结

将这两个关键点结合起来:

1. $\xi_2$ 的贡献为零是因为它与 $\operatorname{Ker}(\bar{\partial})$ 正交,使得它在 $\bar{\partial}$ 的积分中不贡献。
2. $\bar{\partial}\left(\frac{u}{s_Y}\right)$ 在 $Y$ 上有类似 Dirac delta 的表现形式,因此它的积分可以集中在 $Y$ 上的一个拉回,使得积分变得可计算。

这样就得到了从一式到二式的推导。

\Line
要理解这个问题,需要更详细地讨论除子、全纯函数和电流的关系,以及如何在不同维度的子流形上进行积分。

\textbf{背景:Poincaré-Lelong公式和电流}

首先,我们回顾一下\textbf{Poincaré-Lelong公式,它在研究全纯函数的零点集时非常重要}。如果$s_Y$是一个全纯函数,它的零点集定义了一个除子$Y$。Poincaré-Lelong公式表明:
$$
dd^c \log |s_Y|^2 = [Y]
$$
这里,$[Y]$是集中在$Y$上的电流,$dd^c$是复几何中的微分算子。

1. 在$X$上的积分

当我们在$X$上积分时,考虑的是一个包含$Y$的整体空间上的积分。在这个背景下,如果我们有一个形式$u$,则
$$
\bar{\partial} \left(\frac{u}{s_Y}\right)
$$
可以理解为在$X$上去除$Y$时的全纯形式。实际上,这可以通过引入一个小参数$\epsilon$来处理,例如考虑$|s_Y| \geq \epsilon$时的积分。在这种情况下,$\frac{1}{s_Y}$是有意义的。

使用$\bar{\partial} \left(\frac{u}{s_Y}\right)$表示,意味着我们在$X$上处理的是一种带有奇异性(在$Y$上)的全纯形式,它的贡献集中在$Y$附近。

2. 在$Y$上的积分

当我们考虑在$Y$上的积分时,我们实际上是将积分集中在$Y$上,而不是在整个$X$上。为了在$Y$上处理积分,我们需要将形式$\frac{u}{s_Y}$限制到$Y$上。这里的关键是理解$\bar{\partial}\left(\frac{u}{s_Y}\right)$在$Y$上的行为。

使用Poincaré-Lelong公式,我们知道:
$$
\bar{\partial}\left(\frac{1}{s_Y}\right) \text{在} Y \text{上类似于} \delta_Y
$$
这意味着在$Y$上的积分可以看作在一个小邻域内的积分,在这种情况下,形式$\bar{\partial}\left(\frac{u}{s_Y}\right)$实际上变成了$\frac{u}{ds_Y}$,其中$ds_Y$表示$s_Y$在$Y$上的微分。

数学解释

假设$Y$是$X$中的一个全纯子流形,由$s_Y=0$定义。考虑积分:
$$
\int_X \bar{\partial} \left( \frac{u}{s_Y} \right) \wedge \overline{\gamma_{\xi}}
$$
利用Poincaré-Lelong公式,可以写成:
$$
\int_X \bar{\partial} \left( \frac{u}{s_Y} \right) \wedge \overline{\gamma_{\xi}} = \int_Y \left( \frac{u}{ds_Y} \right) \wedge \overline{\gamma_{\xi}}
$$
这种转换是因为$\bar{\partial} \left( \frac{1}{s_Y} \right)$在$Y$上表现为一个delta-like的电流,从而将$X$上的积分集中在$Y$上。

总结:
- $\bar{\partial}\left(\frac{u}{s_Y}\right)$表示在$X$上包含奇异点的全纯形式。
- 在$Y$上,这种形式可以转化为$\frac{u}{ds_Y}$,因为$\bar{\partial}\left(\frac{1}{s_Y}\right)$在$Y$上表现为一个delta-like的电流。

通过这种方法,我们可以从全空间$X$上的积分转换为在子流形$Y$上的积分。要深入理解这种转换和电流理论,推荐参考Demailly的书《Complex Analytic and Differential Geometry》中的相关章节。
\end{remark}

\begin{thm}[Poincar\'e Lelong Equation]\label{thm:plelong} % Demalily (2.15)
    Let $f \in \mathcal{M}(X)$ be a meromorphic function which does not vanish identically on any connected component of $X$ and let $\sum m_j Z_j$ be the divisor of $f$. Then the function $\log |f|$ is locally integrable on $X$ and satisfies the equation
$$
\frac{\mathrm{i}}{\pi} d^{\prime} d^{\prime \prime} \log |f|=\sum m_j\left[Z_j\right]
$$
in the space $\mathscr{D}_{n-1, n-1}^{\prime}(X)$ of currents of bidimension $(n-1, n-1)$.
\end{thm}
Observe that if $f$ is holomorphic, then $\log |f| \in \operatorname{Psh}(X)$, the coefficients $m_j$ are positive integers and the right hand side is a positive current in $\mathscr{D}_{n-1, n-1}^{\prime+}(X)$.


In our setting we have $Y=\bigcup Y_i$, and the difficulty steams from the fact that the functional
$$
\int_Y \frac{u}{\mathrm{~d} s_Y} \wedge \overline{\gamma_{\xi_1}} e^{-\varphi_L}=\sum_i \int_{Y_i} \frac{u}{\prod_{j \neq i} s_j \mathrm{~d} s_{Y_i}} \wedge \overline{\gamma_{\xi_1}} e^{-\varphi_L}
$$
becomes a sum of expressions involving forms with log poles. We have
$$
\frac{u}{\prod_{j \neq i} s_j \mathrm{~d} s_{Y_i}} \notin L^2\left(\left.e^{-\varphi_L}\right|_{Y_i}\right)
$$
in general, so the previous arguments are breaking down.
\begin{problem}[Why?]
Nevertheless we do have
$$
\left|\frac{u}{\mathrm{~d} s_Y}\right|_\omega^{\frac{2}{1+\alpha}} \in L^1\left(Y,\left.\omega\right|_Y\right)
$$
near the singularities of $Y$ for any reasonable metric $\omega$. \textbf{This means that we have to find an estimate of the $L^{\infty}$ norm of $\gamma_{\xi_1} \mid V_{\text {sing } \cap Y}$ in terms of the RHS of (2.0.5).}
\end{problem}

\begin{solution}

\textbf{要理解这里的问题,我们需要分解和细化所提到的各个部分以及它们之间的关系。具体而言,我们要理解的是,为什么在具有简单正则交叉(simple normal crossings, snc)奇异性的情形下,形式$\frac{u}{\mathrm{d} s_Y}$会产生困难,以及这些困难如何影响积分。}

\textbf{背景}
\begin{itemize}
    \item 我们考虑一个复流形$X$及其子流形$Y$,其中$Y$是由几个子流形$Y_i$组成的简单正则交叉(snc)超曲面,即:$Y = \bigcup Y_i$。
\end{itemize}

\textbf{积分表达式及其问题}
\begin{itemize}
    \item 我们关注以下积分:
    \[
    \int_Y \frac{u}{\mathrm{d} s_Y} \wedge \overline{\gamma_{\xi_1}} e^{-\varphi_L}
    \]
    \item 根据定义,这个积分可以被分解成各个$Y_i$上的积分之和:
    \[
    \sum_i \int_{Y_i} \frac{u}{\prod_{j \neq i} s_j \mathrm{d} s_{Y_i}} \wedge \overline{\gamma_{\xi_1}} e^{-\varphi_L}
    \]
\end{itemize}

\textbf{主要问题}
\begin{enumerate}
    \item 在每个子流形$Y_i$上,形式$\frac{u}{\prod_{j \neq i} s_j \mathrm{d} s_{Y_i}}$可能会有对数极点(log poles)。具体来说,对于全纯函数$u$,由于存在多个$s_j$,形式$\frac{u}{\prod_{j \neq i} s_j \mathrm{d} s_{Y_i}}$在$L^2$范数下可能不可积。换句话说:
    \[
    \frac{u}{\prod_{j \neq i} s_j \mathrm{d} s_{Y_i}} \notin L^2\left(\left.e^{-\varphi_L}\right|_{Y_i}\right)
    \]
    \item \textbf{解决方法的思路}
    \begin{itemize}
        \item 尽管如此,我们还是有以下结果:
        \[
        \left|\frac{u}{\mathrm{d} s_Y}\right|_\omega^{\frac{2}{1+\alpha}} \in L^1\left(Y,\left.\omega\right|_Y\right)
        \]
        这意味着,在$Y$的奇异点附近,对于任何合理的度量$\omega$,形式$\frac{u}{\mathrm{d} s_Y}$的某种范数是$L^1$可积的。这给我们提供了一些希望,可以通过某种估计方法来处理这些奇异点。
    \end{itemize}
\end{enumerate}

\textbf{$L^{\infty}$范数的估计}
\begin{itemize}
    \item 最后一步是如何处理$\gamma_{\xi_1}$在$V_{\text{sing} \cap Y}$上的$L^{\infty}$范数。这个估计需要联系到我们之前的表达式,即(2.0.5)的右边。
\end{itemize}

\textbf{详细解释}
\begin{enumerate}
    \item \textbf{分解积分}:
    \begin{itemize}
        \item 对$Y$进行分解,得到每个$Y_i$上的积分。
        \item 注意到在每个$Y_i$上,形式$\frac{u}{\prod_{j \neq i} s_j \mathrm{d} s_{Y_i}}$可能有对数极点。
    \end{itemize}
    \item \textbf{不可积性问题}:
    \begin{itemize}
        \item $\frac{u}{\prod_{j \neq i} s_j \mathrm{d} s_{Y_i}} \notin L^2$意味着积分在这些子流形上不收敛。
    \end{itemize}
    \item \textbf{可积性解决方案}:
    \begin{itemize}
        \item 尽管$L^2$不可积,我们还是可以在某些条件下证明$\left|\frac{u}{\mathrm{d} s_Y}\right|_\omega^{\frac{2}{1+\alpha}} \in L^1$,表明有些更弱的形式是可积的。
    \end{itemize}
    \item \textbf{$L^{\infty}$范数的估计}:
    \begin{itemize}
        \item 目标是找到$\gamma_{\xi_1}$在奇异点的$L^{\infty}$范数的上界,以便控制整体积分。
        \item 这一步的估计需要联系到之前的公式(2.0.5)的右边(未给出具体形式,但假设其提供了必要的界限)。
    \end{itemize}
\end{enumerate}

总结而言,理解这个问题的关键在于处理由snc超曲面$Y$带来的奇异性,特别是形式$\frac{u}{\mathrm{d} s_Y}$的对数极点,并通过某种方式控制这些奇异点对积分的影响。具体方法可以参考相关文献中关于正则交叉点处理的讨论,尤其是在Demailly的书中。

\end{solution}


要证明在给定表达式中 $\left\|\bar{\partial}_{\varphi}^{\star} \xi\right\|^2 - \|\bar{\partial} \xi\|^2$ 是正的($>0$),需要理解 $\bar{\partial}_{\varphi}^{\star} \xi$ 和 $\bar{\partial} \xi$ 之间的关系,并利用适当的数学工具,如Bochner-Kodaira-Nakano恒等式。让我们深入分析这个问题。

1. 相关定义

首先,我们明确以下符号和定义:$\xi$ 是一个 $(n,q)$ 型的微分形式; 
$\bar{\partial}$ 是复流形上的 $\bar{\partial}$ 算子;
$\bar{\partial}_{\varphi}^{\star}$ 是 $\bar{\partial}$ 的伴随算子,它与度量 $\varphi$ 有关;
$\left\|\cdot\right\|^2$ 表示对应形式的 $L^2$ 范数。

2. Bochner-Kodaira-Nakano 恒等式
\begin{thm}[Bochner-Kodaira-Nakano 恒等式]\label{thm:BKN}
    Bochner-Kodaira-Nakano 恒等式是研究 Kähler 流形上复微分形式的一种工具。它的一种形式是:
$$
\sqrt{-1} \partial \bar{\partial} \left\langle \xi, \xi \right\rangle = \left( \left\| \bar{\partial} \xi \right\|^2 + \left\| \partial \xi \right\|^2 - \left\| \bar{\partial}_{\varphi}^{\star} \xi \right\|^2 - \left\| \partial_{\varphi}^{\star} \xi \right\|^2 \right) \mathrm{d} V_\omega + \left\langle \Theta_{\varphi} \xi, \xi \right\rangle,
$$
其中 $\Theta_{\varphi}$ 是由度量 $\varphi$ 引起的曲率形式。
\end{thm}
    
3. 应用到当前情况

在我们的问题中,假设 $\xi$ 是一个 $(n,1)$ 型的形式,Bochner-Kodaira-Nakano 恒等式的一个特例应用如下:
$$
\sqrt{-1} \partial \bar{\partial} \left\langle \xi, \xi \right\rangle = \left( \left\| \bar{\partial} \xi \right\|^2 + \left\| \partial \xi \right\|^2 - \left\| \bar{\partial}_{\varphi}^{\star} \xi \right\|^2 \right) \mathrm{d} V_\omega + \left\langle \Theta_{\varphi} \xi, \xi \right\rangle.
$$

将 $\partial \xi = 0$ 代入,这里 $\left\| \partial \xi \right\|^2 = 0$,则得到:
$$
\sqrt{-1} \partial \bar{\partial} \left\langle \xi, \xi \right\rangle = \left( \left\| \bar{\partial} \xi \right\|^2 - \left\| \bar{\partial}_{\varphi}^{\star} \xi \right\|^2 \right) \mathrm{d} V_\omega + \left\langle \Theta_{\varphi} \xi, \xi \right\rangle.
$$

4. 正性问题

在很多情况下,尤其是当我们讨论某些特定度量(如 Kähler-Einstein 度量或正曲率度量)时,曲率项 $\left\langle \Theta_{\varphi} \xi, \xi \right\rangle$ 是非负的。如果我们考虑 $\sqrt{-1} \partial \bar{\partial} \left\langle \xi, \xi \right\rangle \geq 0$,这表明
$$
\left\| \bar{\partial}_{\varphi}^{\star} \xi \right\|^2 \leq \left\| \bar{\partial} \xi \right\|^2.
$$

因此,如果 $\sqrt{-1} \partial \bar{\partial} T_{\xi}$ 中的项 $\Theta_{h_L}(L) \wedge T_{\xi}$ 的影响很小或者被适当地控制住,那么上述不等式可以转化为 $\left\|\bar{\partial}_{\varphi}^{\star} \xi\right\|^2 - \|\bar{\partial} \xi\|^2$ 为正。

5. 结论

因此,$\left\|\bar{\partial}_{\varphi}^{\star} \xi\right\|^2 - \|\bar{\partial} \xi\|^2$ 是否为正取决于特定情况下的度量和曲率。通过适当选择度量 $\varphi$ 和形式 $\xi$,可以使这个差值为正。


要解释为何公式中的这两项之差$\left\|\bar{\partial}_{\varphi}^{\star} \xi\right\|^2-\|\bar{\partial} \xi\|^2$是正的,我们需要深入理解这些运算符及其范数的定义和性质。

首先,$\xi$是一个$(n,1)$形式。我们将讨论$\bar{\partial}$和$\bar{\partial}_{\varphi}^{\star}$这两个算符的作用及其范数。

定义和性质
1. **Dolbeault算符 $\bar{\partial}$**:
   - 作用在一个$(n,1)$形式$\xi$上,得到一个$(n,2)$形式。
   - 它的形式范数定义为$\|\bar{\partial} \xi\|^2$,这表示在适当的内积下$\bar{\partial} \xi$的范数平方。

2. **$\bar{\partial}_{\varphi}^{\star}$算符**:
   - 这是$\bar{\partial}$的形式伴随算符,作用在一个$(n,1)$形式$\xi$上,得到一个$(n,0)$形式。
   - 它的形式范数定义为$\|\bar{\partial}_{\varphi}^{\star} \xi\|^2$,表示在适当的内积下$\bar{\partial}_{\varphi}^{\star} \xi$的范数平方。

$\bar{\partial}$和$\bar{\partial}_{\varphi}^{\star}$的范数差
为了解释为何$\left\|\bar{\partial}_{\varphi}^{\star} \xi\right\|^2-\|\bar{\partial} \xi\|^2$是正的,我们可以借助Kodaira-Spencer理论的一些结果。特别地,Bochner-Kodaira-Nakano恒等式在这种情况下非常有用。该恒等式用于研究复几何中复结构上的正则形式,给出了如下关系式:
\[
\sqrt{-1} \partial \bar{\partial} T_{\xi} = \left(-2 \Re\left\langle\bar{\partial} \bar{\partial}_{\varphi}^{\star} \xi, \xi\right\rangle + \left\|\bar{\partial} \gamma_{\xi}\right\|^2 + \left\|\bar{\partial}_{\varphi}^{\star} \xi\right\|^2 - \|\bar{\partial} \xi\|^2\right) \mathrm{d}V_\omega + \Theta_{h_L}(L) \wedge T_{\xi}.
\]
这里的一个关键点是理解这个恒等式的应用和$\xi$的几何背景。

Hermitian-Einstein度量和Nakano正性
在许多几何背景下(如当我们处理Hermitian-Einstein度量时),$\bar{\partial}_{\varphi}^{\star} \xi$和$\bar{\partial} \xi$之间的关系反映了某种曲率条件。一般来说,如果$\xi$是一个适当的形式,且它满足某种正性的条件(比如Nakano正性),那么这个恒等式可以帮助揭示$\bar{\partial}_{\varphi}^{\star}$的范数项在一定条件下确实大于$\bar{\partial}$的范数项。

特别地,在这种背景下,$\left\|\bar{\partial}_{\varphi}^{\star} \xi\right\|^2$项会捕捉更多的“正性”,而$\|\bar{\partial} \xi\|^2$则可能包含负性或减少的正性。这种关系使得$\left\|\bar{\partial}_{\varphi}^{\star} \xi\right\|^2$往往会比$\|\bar{\partial} \xi\|^2$更大,从而使得它们的差值是正的。

因此,结合几何背景和算符的性质,这些范数之间的差值$\left\|\bar{\partial}_{\varphi}^{\star} \xi\right\|^2 - \|\bar{\partial} \xi\|^2$在合理假设下可以是正的。


\begin{remark}
    为了详细解释为什么 \(T_\xi\) 和 \(\omega\) 可以在等式 \(\sqrt{-1}\partial\bar{\partial} T_\xi \wedge \omega = \sqrt{-1}\partial\bar{\partial} \omega \wedge T_\xi\) 中互换位置,我们需要理解几何和代数形式上的一些基本性质。具体来说,这涉及到外微分形式的交换性和双线性性质。

1. 外微分形式的楔积
首先,外微分形式的楔积具有反对称性。这意味着对于任何两个微分形式 \(\alpha\) 和 \(\beta\):
\[
\alpha \wedge \beta = (-1)^{\deg(\alpha) +\deg(\beta)} \beta \wedge \alpha
\]

其中 \(\deg(\alpha)\) 和 \(\deg(\beta)\) 分别是 \(\alpha\) 和 \(\beta\) 的阶数(degree)。如果 \(\alpha\) 和 \(\beta\) 都是 1-形式,\(\alpha \wedge \beta = -\beta \wedge \alpha\)。

2. 双线性性质
楔积是双线性的,即对于任何三个微分形式 \(\alpha\)、\(\beta\) 和 \(\gamma\),我们有:
\[
\alpha \wedge (\beta + \gamma) = \alpha \wedge \beta + \alpha \wedge \gamma
\]

3. 外微分运算
外微分运算符 \(\partial\) 和 \(\bar{\partial}\) 满足莱布尼兹法则。这意味着,对于任何两个微分形式 \(\alpha\) 和 \(\beta\),我们有:
\[
\partial(\alpha \wedge \beta) = (\partial \alpha) \wedge \beta + (-1)^{\deg(\alpha)} \alpha \wedge (\partial \beta)
\]
\[
\bar{\partial}(\alpha \wedge \beta) = (\bar{\partial} \alpha) \wedge \beta + (-1)^{\deg(\alpha)} \alpha \wedge (\bar{\partial} \beta)
\]

4. 等式的对称性
现在考虑等式 \(\sqrt{-1}\partial\bar{\partial} T_\xi \wedge \omega = \sqrt{-1}\partial\bar{\partial} \omega \wedge T_\xi\)。由于我们在计算过程中使用的是外微分和楔积运算,这些运算具有良好的对称性和双线性性质。

具体来说,当我们计算 \(\partial\) 和 \(\bar{\partial}\) 时,形式之间的楔积不影响结果,因为:
\[
\partial(T_\xi \wedge \omega) = (\partial T_\xi) \wedge \omega + (-1)^{\deg(T_\xi)} T_\xi \wedge (\partial \omega)
\]
类似地,
\[
\bar{\partial}(T_\xi \wedge \omega) = (\bar{\partial} T_\xi) \wedge \omega + (-1)^{\deg(T_\xi)} T_\xi \wedge (\bar{\partial} \omega)
\]

5. 互换位置的推导
通过对称性,我们有:
\[
\sqrt{-1} \partial \bar{\partial} (T_\xi \wedge \omega) = \sqrt{-1} (\partial (\bar{\partial} T_\xi) \wedge \omega + (-1)^{\deg(T_\xi)} \bar{\partial} T_\xi \wedge \partial \omega)
\]

根据微分形式的交换性和外微分的性质,这可以等效地写成:
\[
\sqrt{-1} \partial \bar{\partial} T_\xi \wedge \omega = \sqrt{-1} \partial \bar{\partial} \omega \wedge T_\xi
\]

因为楔积是双线性的,所以形式在楔积中的相对位置不影响外微分运算的结果。这种对称性确保了 \(T_\xi\) 和 \(\omega\) 可以互换位置。因此,
\[
\sqrt{-1}\partial\bar{\partial} T_\xi \wedge \omega = \sqrt{-1}\partial\bar{\partial} \omega \wedge T_\xi
\]

是成立的。
\end{remark}


