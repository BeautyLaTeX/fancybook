\documentclass[lang=cn,zihao=-4,fontset=none,twoside]{fancybook}
\usepackage{zhlipsum}
\usepackage{amssymb}
\let\heavymath\relax
\let\Bbbk\relax
\usepackage{mtpro2}
\DeclareSymbolFont{CMlargesymbols}{OMX}{cmex}{m}{n}
\let\sum\relax
\let\prod\relax
\DeclareMathSymbol{\sum}{\mathop}{CMlargesymbols}{"50}
\DeclareMathSymbol{\prod}{\mathop}{CMlargesymbols}{"51}
%% \infty
\DeclareSymbolFont{symbolsCM}{OMS}{cmsy}{m}{n}
\SetSymbolFont{symbolsCM}{bold}{OMS}{cmsy}{b}{n}

\DeclareMathSymbol{\infty}{\mathord}{symbolsCM}{"31}
\let\dbar\relax
\DeclareMathOperator*{\res}{Res}
\newcommand{\pr}{^\prime}
\newcommand{\prr}{^{\prime\prime}}
\newcommand{\bd}{\partial}
\newcommand{\bdd}{\bar\partial}
\newcommand{\Dif}[2][]{\frac{\dd #1}{\dd #2}}
\newcommand{\Diff}[2][]{\frac{\partial #1}{\partial #2}}
\newcommand{\dbar}{\bar\partial}
\newcommand{\supp}{\operatorname{Supp}}
\newcommand{\pa}{\partial}
\newcommand{\ov}{\overline}
\newcommand{\xu}{\sqrt{-1}}
\newcommand{\lang}[1]{\langle #1\rangle}
\newcommand{\itbf}[1]{\textit{\textbf{#1}}}
% 定义紧密双内积符号
\newcommand{\llangle}{\left\langle\!\left\langle}
\newcommand{\rrangle}{\right\rangle\!\right\rangle}
\usepackage{booktabs}
\usepackage{shadowtext}\shadowoffset{.65pt}
\newcommand{\Line}{\noindent\tikz\draw[line width=0.65pt,gray!80,dashed] (0,0)--++(.99\linewidth,0);\par}
\newenvironment{note}[1][\bf Note:]{\par\Line\uuline{#1} }{\par\Line}




\usetikzlibrary{patterns,shapes.geometric,3d,shadings,shadows} % 加载必要库

% 定义金色颜色
\definecolor{gold}{RGB}{255,215,0}
\definecolor{parchment}{RGB}{245,245,220} % 定义淡黄色背景
\definecolor{deepgold}{HTML}{e7ded5}
% 定义交叉网格模式
\pgfdeclarepatternformonly{customgrid}
{\pgfpointorigin}
{\pgfpoint{3pt}{3pt}}
{\pgfpoint{3pt}{3pt}}
{
    \pgfsetlinewidth{0.3pt}
    \pgfpathmoveto{\pgfpoint{0pt}{0pt}}
    \pgfpathlineto{\pgfpoint{3pt}{3pt}}
    \pgfpathmoveto{\pgfpoint{0pt}{3pt}}
    \pgfpathlineto{\pgfpoint{3pt}{0pt}}
    \pgfusepath{stroke}
}

\pgfdeclarelayer{background} %背景%底层
\pgfdeclarelayer{foreground} %上层
\pgfdeclarelayer{top} %顶部
\pgfdeclarelayer{bottom} %底部
\pgfsetlayers{bottom,background,main,foreground,top}

\makeatletter
% ---------------------------------------------------------------------------------- %
% /* ---------------------------------- 第二种定理 --------------------------------- */
\newtcolorbox{test}{
	breakable,
	enhanced,
	sharp corners,
	boxsep=1pt,
	top=1ex,
	boxrule=0pt,
	colframe=deepgold,
	colback=deepgold!20!white,
	overlay unbroken and first={
		\my@theorem@overlay@unbroken{gray}
}}

%%  Overlay Settings
\newcommand{\my@theorem@overlay@unbroken}[1]{
% 左侧圆柱体
\node[cylinder, 
draw=#1, 
cylinder uses custom fill,
cylinder body fill=parchment, 
cylinder end fill=parchment!70,
minimum width=0.5cm, % 更窄的圆柱体宽度
minimum height=8cm,
shape border rotate=90, % 调整方向
anchor=center,drop shadow={opacity=0.3,shadow xshift=0.3mm,shadow yshift=-0.5mm},] (left) at ([yshift=-3cm]frame.north west) {};

% 左侧圆柱体花纹
\begin{scope}
  \clip ([shift={(-0.25,0)}]frame.north west) rectangle  +(0.5,0); % 更窄的花纹范围
  \fill[pattern=customgrid, pattern color=gray!40] ([shift={(-0.25,0)}]frame.north west) rectangle  +(0.5,0);
\end{scope}

% 右侧圆柱体
\node[cylinder, 
draw=gray, 
cylinder uses custom fill,
cylinder body fill=parchment, 
cylinder end fill=parchment!70,
minimum width=0.5cm, % 更窄的圆柱体宽度
minimum height=6cm,
shape border rotate=90, % 调整方向
anchor=center,drop shadow={opacity=0.3,shadow xshift=-0.3mm,shadow yshift=-0.5mm},] (right) at (frame.north east) {};

% 右侧圆柱体花纹
\begin{scope}
  \clip ([shift={(0.25,0)}]frame.north east) rectangle  +(-0.5,0); % 更窄的花纹范围
  \fill[pattern=customgrid, pattern color=gray!40] ([shift={(0.25,0)}]frame.north east) rectangle  +(-0.5,0);
\end{scope}


% 左侧金属球(上下两个,与圆柱体贴合)
\begin{pgfonlayer}{bottom}
\shade[ball color=gold] (frame.north west) circle (0.3); % 更大的金属球
\end{pgfonlayer}
%  \begin{pgfonlayer}{foreground}
\shade[ball color=gold] (frame.south west) circle (0.3);
% \end{pgfonlayer}


% 右侧金属球(上下两个,与圆柱体贴合)
\begin{pgfonlayer}{bottom}
\shade[ball color=gold] (frame.north east) circle (0.3); % 更大的金属球
\end{pgfonlayer}
\shade[ball color=gold] (frame.south east) circle (0.3);

}
% \newcommand{\my@theorem@overlay@first}[2]{

% }
% \newcommand{\my@theorem@overlay@last}[1]{

% }


\makeatother

\begin{document}
\thispagestyle{empty}
\title{课题调研与选题探讨}
\subtitle{上同调学习笔记}
\edition{第一版}
\bookseries{课\hspace{.5em}题\hspace{.5em}调\hspace{.5em}研}
\author{Shilong.Lu}
\pressname{Springer} %出版社名称
\presslogo{figures/Springer-logo.png} %出版社徽标
\makecover
\input{frontmatter/titlepage}
\frontmatter
\pagenumbering{Roman}
\input{frontmatter/preface}
\thispagestyle{empty}
\definecolor{maincolor}{HTML}{8B8989}
\tableofcontents\let\cleardoublepage\clearpage

\mainmatter
\pagenumbering{arabic}
\partimage{figures/flower.png}
\partabstract{}
\part{Blow-up 与Dolbeault 上同调学习笔记}
\chapter{问题与回答集锦}
\section{一些问答合集}
\subsection{问题与回答}
\[
b_1^q + b_2^q + \cdots + b_d^q = \frac{1}{2\pi \sqrt{-1}} \int_{|w|=r} \frac{w^q \frac{\partial f}{\partial w}(z, w)}{f(z, w)} dw.
\]

这个公式的推导基于复变函数的 \textbf{留数定理}
和解析函数的性质。下面逐步解释推导的过程:

\begin{center}\rule{0.5\linewidth}{0.5pt}\end{center}

\hypertarget{ux80ccux666fux548cux76eeux6807}{%
\subsubsection{背景和目标}\label{ux80ccux666fux548cux76eeux6807}}

函数 \(f(z, w)\) 是 \(z, w\) 的复变量函数,在 \(w\) 上的零点(根)为
\(b_1, b_2, \dots, b_d\)。目标是通过积分公式计算与这些零点相关的某些性质(例如零点的幂和等)。

\begin{center}\rule{0.5\linewidth}{0.5pt}\end{center}

\hypertarget{ux96f6ux70b9ux7684ux6027ux8d28ux548cux7559ux6570ux5b9aux7406}{%
\subsubsection{1.
零点的性质和留数定理}\label{ux96f6ux70b9ux7684ux6027ux8d28ux548cux7559ux6570ux5b9aux7406}}

根据 \(f(z, w)\) 在 \(w\) 上的零点性质,假设
\(w = b_1, b_2, \dots, b_d\) 是 \(f(z, w) = 0\) 在 \(w\)
上的零点,且这些零点位于单位圆盘 \$ \textbar{}w\textbar{} \textless{} r
\$ 内。

\hypertarget{ux7559ux6570ux5b9aux7406ux7684ux5e94ux7528}{%
\paragraph{留数定理的应用}\label{ux7559ux6570ux5b9aux7406ux7684ux5e94ux7528}}

根据复分析,假设 \(f(z, w)\) 在 \(w\)
上的零点为简单零点(重零点的情况需要调整权重),则积分:

\[
\frac{1}{2\pi \sqrt{-1}} \int_{|w|=r} \frac{w^q \frac{\partial f}{\partial w}(z, w)}{f(z, w)} dw
\]

等于 \(w = b_1, b_2, \dots, b_d\) 处的留数之和。

\begin{center}\rule{0.5\linewidth}{0.5pt}\end{center}

\hypertarget{ux7559ux6570ux8ba1ux7b97ux4e0eux96f6ux70b9ux5c55ux5f00}{%
\subsubsection{2.
留数计算与零点展开}\label{ux7559ux6570ux8ba1ux7b97ux4e0eux96f6ux70b9ux5c55ux5f00}}

\hypertarget{ux5355ux4e2aux96f6ux70b9ux7684ux8d21ux732e}{%
\paragraph{单个零点的贡献}\label{ux5355ux4e2aux96f6ux70b9ux7684ux8d21ux732e}}

对于 \(f(z, w)\) 在 \(w = b_i\) 的简单零点,可以写成:

\[
f(z, w) = (w - b_i) f'(z, b_i) + \text{高阶项} \quad \text{(忽略高阶项)}.
\]

因此,\(\frac{\partial f}{\partial w}(z, w)\) 在 \(w = b_i\) 近似为
\(f'(z, b_i)\),且积分中 \(w^q\) 项在零点处的贡献为:

\[
    \text{Residue at } w = b_i: \quad \frac{w^q}{f'(z, b_i)}.
\]

\hypertarget{ux6240ux6709ux96f6ux70b9ux7684ux8d21ux732e}{%
\paragraph{所有零点的贡献}\label{ux6240ux6709ux96f6ux70b9ux7684ux8d21ux732e}}

对所有零点 \(b_1, b_2, \dots, b_d\),积分公式累加留数:

\[
\frac{1}{2\pi \sqrt{-1}} \int_{|w|=r} \frac{w^q \frac{\partial f}{\partial w}(z, w)}{f(z, w)} dw = b_1^q + b_2^q + \cdots + b_d^q.
\]

这里 \(b_1^q + b_2^q + \cdots + b_d^q\) 是所有零点的 \(q\) 次幂的和。

\begin{center}\rule{0.5\linewidth}{0.5pt}\end{center}

\hypertarget{ux7ed3ux679cux89e3ux91ca}{%
\subsubsection{3. 结果解释}\label{ux7ed3ux679cux89e3ux91ca}}

公式中的积分与零点 \(b_1, b_2, \dots, b_d\)
的关系直接来源于留数定理。通过这个积分形式,可以解析地计算出零点的 \(q\)
次幂的和。这种方法在解析函数论和几何中非常有用。


要推导你提出的公式:

\[
\sum_{j=1}^d b_j^q = \frac{1}{2\pi i} \int_{|w|=r} \frac{w^q \left(\frac{\partial f(z, w)}{\partial w}\right)}{f(z, w)} \, dw,
\]





\textbf{为何积分函数选择这种形式},以及 \textbf{留数为何会变成
\(b_i^q\)}。

\begin{center}\rule{0.5\linewidth}{0.5pt}\end{center}

\hypertarget{ux4e00ux4e3aux4f55ux79efux5206ux51fdux6570ux662f-fracwq-fracpartial-fpartial-wz-wfz-w}{%
\subsubsection{\texorpdfstring{一、为何积分函数是
\(\frac{w^q \frac{\partial f}{\partial w}(z, w)}{f(z, w)}\)?}{一、为何积分函数是 \textbackslash{}frac\{w\^{}q \textbackslash{}frac\{\textbackslash{}partial f\}\{\textbackslash{}partial w\}(z, w)\}\{f(z, w)\}?}}\label{ux4e00ux4e3aux4f55ux79efux5206ux51fdux6570ux662f-fracwq-fracpartial-fpartial-wz-wfz-w}}

首先回顾留数定理的基本思想:对于一个解析函数 \(f(w)\),如果它在
\(w = b_i\) 有一个 \textbf{简单零点},那么其 \textbf{留数}
可以由下式计算:

\[
\text{Residue at } w = b_i: \quad \text{Res}_{w = b_i}\left(\frac{g(w)}{f(w)}\right) = \frac{g(b_i)}{f'(b_i)},
\]

这里 \(g(w)\) 是另一个解析函数,且 \(f(b_i) = 0\) 是简单零点。

\begin{center}\rule{0.5\linewidth}{0.5pt}\end{center}

在这个公式中,我们的目标是将零点 \(b_1, b_2, \dots, b_d\)
的信息提取出来,并集中在积分形式中。

\hypertarget{ux5173ux4e8eux79efux5206ux51fdux6570ux7684ux9009ux62e9}{%
\paragraph{1.
关于积分函数的选择}\label{ux5173ux4e8eux79efux5206ux51fdux6570ux7684ux9009ux62e9}}

目标是通过积分公式构造出 \textbf{零点的 \(q\) 次幂和},即
\(b_1^q + b_2^q + \cdots + b_d^q\)。所以积分核 \(\frac{g(w)}{f(z, w)}\)
中的 \(g(w)\) 应该选择成一个 \textbf{与零点 \(b_i\) 相关的函数}。

为了提取零点的 \(q\) 次幂,选择 \(g(w) = w^q\)
是一个自然的选择,因为零点 \(b_i\) 的贡献将是 \(w = b_i\) 处的 \(w^q\)
值。

\hypertarget{ux51fdux6570-fracpartial-fpartial-wz-w-ux7684ux5f15ux5165}{%
\paragraph{\texorpdfstring{2. 函数
\(\frac{\partial f}{\partial w}(z, w)\)
的引入}{2. 函数 \textbackslash{}frac\{\textbackslash{}partial f\}\{\textbackslash{}partial w\}(z, w) 的引入}}\label{ux51fdux6570-fracpartial-fpartial-wz-w-ux7684ux5f15ux5165}}

如果 \(f(z, w)\) 在 \(w = b_i\) 有一个零点,局部展开为:
\[
f(z, w) = (w - b_i) f'(z, b_i) + \text{高阶项}.
\]

这里假设 \(b_i\) 是 \textbf{简单零点},因此零点处的导数不为零,即
\(f'(z, b_i) \neq 0\)。

对于留数计算,需要 \(\frac{\partial f}{\partial w}(z, w)\)
出现在分子中,因为这样才能在积分中抵消 \(f(z, w)\) 在零点处的一阶行为
\((w - b_i)\):
\[
\frac{w^q \frac{\partial f}{\partial w}(z, w)}{f(z, w)} \quad \text{在 } w = b_i \text{ 附近的主要贡献是:} \quad \frac{b_i^q}{f'(z, b_i)}.
\]

这使得留数的贡献只和零点的 \(q\) 次幂 \(b_i^q\) 相关。

\begin{center}\rule{0.5\linewidth}{0.5pt}\end{center}

\hypertarget{ux4e8cux4e3aux4f55ux7559ux6570ux53d8ux6210ux4e86-b_iq}{%
\subsubsection{\texorpdfstring{二、为何留数变成了
\(b_i^q\)?}{二、为何留数变成了 b\_i\^{}q?}}\label{ux4e8cux4e3aux4f55ux7559ux6570ux53d8ux6210ux4e86-b_iq}}

\hypertarget{ux7559ux6570ux7684ux8ba1ux7b97}{%
\paragraph{1. 留数的计算}\label{ux7559ux6570ux7684ux8ba1ux7b97}}

在 \(w = b_i\) 附近,将 \(f(z, w)\) 局部展开为:
\[
f(z, w) \approx (w - b_i) f'(z, b_i),
\]
         
代入积分核
\(\frac{w^q \frac{\partial f}{\partial w}(z, w)}{f(z, w)}\),我们得:
\[
\frac{w^q \frac{\partial f}{\partial w}(z, w)}{f(z, w)} \approx \frac{w^q f'(z, w)}{(w - b_i) f'(z, b_i)}.
\]

在 \(w = b_i\) 附近,\(f'(z, w) \approx f'(z, b_i)\),所以进一步化简为:
\[
\frac{w^q \frac{\partial f}{\partial w}(z, w)}{f(z, w)} \approx \frac{w^q}{w - b_i}.
\]

此时只剩下经典的 \textbf{留数积分形式},即:
\[
\text{Res}_{w = b_i} \left(\frac{w^q}{w - b_i}\right) = b_i^q.
\]

\hypertarget{ux5168ux79efux5206ux7684ux7ed3ux679c}{%
\paragraph{2. 全积分的结果}\label{ux5168ux79efux5206ux7684ux7ed3ux679c}}

由于积分绕的是一个包含所有零点 \(b_1, b_2, \dots, b_d\) 的圈
\(|w| = r\),而积分中每个零点的贡献是它的留数之和,因此最终得到:
\[
\frac{1}{2\pi \sqrt{-1}} \int_{|w| = r} \frac{w^q \frac{\partial f}{\partial w}(z, w)}{f(z, w)} dw = b_1^q + b_2^q + \cdots + b_d^q.
\]

\begin{center}\rule{0.5\linewidth}{0.5pt}\end{center}

\hypertarget{ux4e09ux603bux7ed3}{%
\subsubsection{三、总结}\label{ux4e09ux603bux7ed3}}

\begin{enumerate}
\def\labelenumi{\arabic{enumi}.}
\item
  \textbf{积分函数的形式}:选择
  \(\frac{w^q \frac{\partial f}{\partial w}(z, w)}{f(z, w)}\)
  是为了匹配留数的公式,同时提取零点的 \(q\) 次幂信息。
\item
  \textbf{留数为何是 \(b_i^q\)}:通过将 \(f(z, w)\)
  在零点附近展开,积分核 \(\frac{w^q}{w - b_i}\) 的留数恰好是
  \(b_i^q\)。
\end{enumerate}



  

% \begin{test}
%   \zhlipsum[1]
% \end{test}
  









\normalem
% \printbibliography[
% heading=bibintoc,
% title={参考文献}
% ]
% \printindex

\ISBNcode{\EANisbn[ISBN=978-80-7340-097-2]} %
\summary{这是收集了一些杂七杂八的资料的笔记。}
\makebottomcover
\end{document}